\documentclass[a4paper]{article}

%% Language and font encodings
\usepackage[english]{babel}
\usepackage[utf8x]{inputenc}
\usepackage[T1]{fontenc}

%% Sets page size and margins
\usepackage[a4paper,top=3cm,bottom=2cm,left=3cm,right=3cm,marginparwidth=1.75cm]{geometry}

%% Useful packages
\usepackage{amsmath}
\usepackage{graphicx}
\usepackage{listings}
\usepackage[colorinlistoftodos]{todonotes}
\usepackage[colorlinks=true, allcolors=blue]{hyperref}

\title{Your Paper}
\author{You}

\begin{document}
\maketitle

\begin{abstract}
	Your abstract.
\end{abstract}

\section{Introduction}

Your introduction goes here! Some examples of commonly used commands and features are listed below, to help you get started. If you have a question, please use the help menu (``?'') on the top bar to search for help or ask us a question. 

\section{Some examples to get started}

\subsection{How to include Figures}

First you have to upload the image file from your computer using the upload link the project menu. Then use the includegraphics command to include it in your document. Use the figure environment and the caption command to add a number and a caption to your figure. See the code for Figure in this section for an example.

\section{Code}
\subsection{complex.h}
\noindent\begin{minipage}{.45\textwidth}
\begin{lstlisting}[caption=code 1 1,frame=tlrb, language=C]{Name}
struct complex
{
double r;
double i;
};

struct complex16
{
short r;
short i;
};

struct complex32
{
int r;
int i;
};
\end{lstlisting}
\end{minipage}\hfill
\begin{minipage}{.45\textwidth}
\begin{lstlisting}[caption=code 2,frame=tlrb, language=C]{Name}
struct complex{
double r;
double i;
};

struct complex16{
int16_t r;
int16_t i;
};

struct complex32{
int32_t r;
int32_t i;
};
\end{lstlisting}
\end{minipage}




\subsection{fft.c}
\begin{lstlisting}[caption=code 3,frame=tlrb, language=C]{Name}
void twiddle(struct complex *W, int N, double stuff){
W->r=cos(stuff*2.0*PI/(double)N);
W->i=-sin(stuff*2.0*PI/(double)N);
}

void twiddle_fixed(struct complex16 *W, int N, double stuff){
W->r=(int16_t)(32767.0*cos(stuff*2.0*PI/(double)N));
W->i=(int16_t)(-32768.0*sin(stuff*2.0*PI/(double)N));
}

void twiddle_fixed_Q17(struct complex32 *W, int N, double stuff){
W->r=(int32_t)(131071.0*cos(stuff*2.0*PI/(double)N));
W->i=(int32_t)(-131072.0*sin(stuff*2.0*PI/(double)N));
}
\end{lstlisting}

\noindent\begin{minipage}{.45\textwidth}
\begin{lstlisting}[caption=code 4,frame=tlrb, language=C]{Name}
void bit_r4_reorder_fixed_Q15(
struct complex16 *W, 
int N,
char scale)
{
int bits, i, j, k;
int16_t tempr, tempi;

for(i=0; i<N; i++){
  W[i].r=W[i].r>>scale;
  W[i].i=W[i].i>>scale;
}

for (i=0; i<MAXPOW; i++)
if (pow_2[i]==N) bits=i;

for (i=0; i<N; i++){
  j=0;
  for (k=0; k<bits; k+=2){
    ...
  }

if (j>i){
  tempr=W[i].r;
  tempi=W[i].i;
  W[i].r=W[j].r;
  W[i].i=W[j].i;
  W[j].r=tempr;
  W[j].i=tempi;
}
}
}
\end{lstlisting}
\end{minipage}\hfill
\begin{minipage}{.45\textwidth}
\begin{lstlisting}[caption=code 5,frame=tlrb, language=C]{Name}
void bit_r4_reorder_fixed_Q15(
struct complex16 *W, 
int N,
char scale)
{
int bits, i, j, k;
short tempr, tempi;







for (i=0; i<MAXPOW; i++)
if (pow_2[i]==N) bits=i;

for (i=0; i<N; i++){
  j=0;
  for (k=0; k<bits; k+=2){
    ...
  }

if (j>i){
  tempr=W[i].r>>scale;
  tempi=W[i].i>>scale;
  W[i].r=W[j].r>>scale;
  W[i].i=W[j].i>>scale;
  W[j].r=tempr;
  W[j].i=tempi;
}
}
}

\end{lstlisting}
\end{minipage}

\begin{lstlisting}[caption=code 5,frame=tlrb, language=C]{Name}
void radix4_fixed_Q15(struct complex16 *x,   // Input in Q15 format 
int N,                 // Size of FFT
unsigned char *scale,  // Pointer to scaling schedule
unsigned char stage)   // Stage of fft
{ 
int    n2, k1, N1, N2;
struct complex16 W, bfly[4];

N1=4;
N2=N/4;

// Do 4 Point DFT  
for (n2=0; n2<N2; n2++){
// scale Butterfly input
x[n2].r          >>= scale[stage];
x[N2+n2].r       >>= scale[stage];
x[(2*N2) + n2].r >>= scale[stage];
x[(3*N2) + n2].r >>= scale[stage];
x[n2].i          >>= scale[stage];
x[N2+n2].i       >>= scale[stage];
x[(2*N2) + n2].i >>= scale[stage];
x[(3*N2) + n2].i >>= scale[stage];

// Radix 4 Butterfly 
bfly[0].r = SAT_ADD16(  SAT_ADD16(x[n2].r, x[N2 + n2].r) , 
SAT_ADD16(x[2*N2+n2].r, x[3*N2+n2].r) 
);
bfly[0].i = SAT_ADD16(  SAT_ADD16(x[n2].i, x[N2 + n2].i) , 
SAT_ADD16(x[2*N2+n2].i, x[3*N2+n2].i) 
);
bfly[1].r = SAT_ADD16(  SAT_ADD16(x[n2].r, x[N2 + n2].i) , 
-SAT_ADD16(x[2*N2+n2].r, x[3*N2+n2].i) 
);
bfly[1].i = SAT_ADD16(  SAT_ADD16(x[n2].i, -x[N2 + n2].r) , 
SAT_ADD16(-x[2*N2+n2].i, x[3*N2+n2].r)
);
bfly[2].r = SAT_ADD16(  SAT_ADD16(x[n2].r, -x[N2 + n2].r) , 
SAT_ADD16(x[2*N2+n2].r, -x[3*N2+n2].r)
);
bfly[2].i = SAT_ADD16(  SAT_ADD16(x[n2].i, -x[N2 + n2].i) , 
SAT_ADD16(x[2*N2+n2].i, -x[3*N2+n2].i)
);
bfly[3].r = SAT_ADD16(  SAT_ADD16(x[n2].r, -x[N2 + n2].i) ,
SAT_ADD16(-x[2*N2+n2].r, x[3*N2+n2].i)
);
bfly[3].i = SAT_ADD16(  SAT_ADD16(x[n2].i, x[N2 + n2].r) ,
SAT_ADD16(-x[2*N2+n2].i, -x[3*N2+n2].r)
);

// In-place results 
x[n2].r = bfly[0].r;
x[n2].i = bfly[0].i;

for (k1=1; k1<N1; k1++){
twiddle_fixed(&W, N, (double)k1*(double)n2);
x[n2 + N2*k1].r = SAT_ADD16( FIX_MPY(bfly[k1].r, W.r) , 
-FIX_MPY(bfly[k1].i, W.i) );
x[n2 + N2*k1].i = SAT_ADD16( FIX_MPY(bfly[k1].i, W.r) , 
FIX_MPY(bfly[k1].r, W.i) );
}
}

// Don't recurse if we're down to one butterfly 
if (N2!=1)
for (k1=0; k1<N1; k1++){
radix4_fixed_Q15(&x[N2*k1], N2,scale,stage+1);
}
}
\end{lstlisting}

Comments can be added to your project by clicking on the comment icon in the toolbar above. % * <john.hammersley@gmail.com> 2016-07-03T09:54:16.211Z:
%
% Here's an example comment!
%
To reply to a comment, simply click the reply button in the lower right corner of the comment, and you can close them when you're done.

Comments can also be added to the margins of the compiled PDF using the todo command\todo{Here's a comment in the margin!}, as shown in the example on the right. You can also add inline comments:

\todo[inline, color=green!40]{This is an inline comment.}

\subsection{How to add Tables}

Use the table and tabular commands for basic tables --- see Table~\ref{tab:widgets}, for example. 

\begin{table}
\centering
\begin{tabular}{l|r}
Item & Quantity \\\hline
Widgets & 42 \\
Gadgets & 13
\end{tabular}
\caption{\label{tab:widgets}An example table.}
\end{table}

\subsection{How to write Mathematics}

\LaTeX{} is great at typesetting mathematics. Let $X_1, X_2, \ldots, X_n$ be a sequence of independent and identically distributed random variables with $\text{E}[X_i] = \mu$ and $\text{Var}[X_i] = \sigma^2 < \infty$, and let
\[S_n = \frac{X_1 + X_2 + \cdots + X_n}{n}
= \frac{1}{n}\sum_{i}^{n} X_i\]
denote their mean. Then as $n$ approaches infinity, the random variables $\sqrt{n}(S_n - \mu)$ converge in distribution to a normal $\mathcal{N}(0, \sigma^2)$.


\subsection{How to create Sections and Subsections}

Use section and subsections to organize your document. Simply use the section and subsection buttons in the toolbar to create them, and we'll handle all the formatting and numbering automatically.

\subsection{How to add Lists}

You can make lists with automatic numbering \dots

\begin{enumerate}
	\item Like this,
	\item and like this.
\end{enumerate}
\dots or bullet points \dots
\begin{itemize}
	\item Like this,
	\item and like this.
\end{itemize}

\subsection{How to add Citations and a References List}

You can upload a \verb|.bib| file containing your BibTeX entries, created with JabRef; or import your \href{https://www.overleaf.com/blog/184}{Mendeley}, CiteULike or Zotero library as a \verb|.bib| file. You can then cite entries from it, like this: \cite{greenwade93}. Just remember to specify a bibliography style, as well as the filename of the \verb|.bib|.

You can find a \href{https://www.overleaf.com/help/97-how-to-include-a-bibliography-using-bibtex}{video tutorial here} to learn more about BibTeX.

We hope you find Overleaf useful, and please let us know if you have any feedback using the help menu above --- or use the contact form at \url{https://www.overleaf.com/contact}!

\bibliographystyle{alpha}
\bibliography{sample}

\end{document}
